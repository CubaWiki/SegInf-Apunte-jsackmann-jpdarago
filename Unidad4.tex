\section{Unidad 4}

\subsection{Introducción}

\begin{itemize}
	\item Definiciones: 
	\begin{itemize}
		\item Autenticación: Asociación entre una identidad y un objeto.
		\item Maneras de establecer identidad: \begin{itemize}
			\item Que es lo que la entidad sabe.
			\item Que es lo que la entidad tiene.
			\item Que es lo que la entidad es físicamente (e.g., huellas digitales).
			\item Donde esta la entidad.
		\end{itemize}
		\item Sistema de Autenticación: Tupla $(A,C,F,L,S)$ con: \begin{itemize}
			\item $A$ es la información que prueba la identidad.
			\item $C$ información almacenada digitalmente para validar la identidad.
			\item $F$ función que transforma $A$ en $C$.
			\item $L$ función que prueba la identidad.
			\item $S$ un conjunto de funciones que permiten crear o alterar información en $A$ o $C$.
		\end{itemize}
		\item Ejemplo para sistema de almacenamiento de claves en texto plano:
		\begin{itemize}
			\item $A$: conjunto de strings que forman claves.
			\item $C$: $A$
			\item $F$: función identidad.
			\item $L$: función de comparación de strings.
			\item $S$: funciones para editar el archivo con las claves.
		\end{itemize}
		\item Claves: \begin{itemize}
			\item Secuencias de caracteres o palabras.
			\item Algoritmos como one-time passwords o challenge response.
		\end{itemize}
		\item Almacenamiento de claves \begin{itemize}
			\item Texto transparente: Inseguro, si se compromete el archivo se comprometen todas las claves.
			\item Archivo cifrado: Necesita claves de cifrado y descifrado, terminamos en problema anterior.
			\item One way hash de la clave: Si se compromete el archivo, el atacante debe invertir la función
			de one way hash, o adivinar las claves.
		\end{itemize}
		\item Salt: comprende bits aleatorios que  son usados como una de las entradas en una función derivadora de claves. 
			La otra entrada es habitualmente una contraseña. La salida de la función derivadora de  claves se almacena como 
			la versión cifrada de la contraseña. La sal puede también ser usada como parte de una clave en un cifrado u otro 
			algoritmo criptográfico. La función de derivación de claves generalmente usa una función hash.
	\end{itemize}
\end{itemize}

\subsection{Sistemas de autentificación}

\subsubsection{UNIX Passwords}

\begin{itemize}
	\item Usa hash de la clave, usando una función de hash dentro de 4096 para pasarlo a un string de 11 caracteres.
	\item Sistema: \begin{itemize}
		\item $A$: Strings de 8 o menos caracteres.
		\item $C$: 2 letras de id de hash concatenadas a 11 letras del hash en si.
		\item $F$: 4096 versiones de DES modificadas.
		\item $L$: login, su
		\item $S$: passwd, nispasswd, passwd+
	\end{itemize}
	\item Algoritmo \begin{itemize}
		\item Cifra un bloque de 64 bits en 0 con DES y el password como clave.
		\item Repite esto 25 veces usando como clave en cada paso el resultado de la anterior.
		\item Se selecciona la variante de DES usando un SALT con la hora del día. Esta se almacena sin cifrar junto
		con el password.
	\end{itemize}
\end{itemize}

\subsubsection{Ataques}

\begin{itemize}
	\item Ataque: Tiene como objetivo encontrar $f \in F$, $f(a) = c \in C$ y que $c$ este asociado a una entidad.
		\begin{itemize}
			\item Se puede lograr directamente o indirectamente (mediante un $l(a)$ tal que $f(l(a)) = c$).
		\end{itemize}
	\item Prevencion: \begin{itemize}
		\item Ocultar la mayor cantidad de variables posible.
		\item Bloquear acceso a $l \in L$ o al resultado de $l(a)$, por ejemplo anulando login desde la red.
	\end{itemize}
	\item Ataques posibles
	\begin{itemize}
		\item Diccionario: Consiste en adivinar probando todas las palabras de una lista. Se puede hacer offline
		si conocemos $f$ y $c$ y probamos multiples $a$ hasta que coincida. Sino, hacemos intentos de login (mas lento).
		\item Fuerza bruta: Probar todas las combinaciones posibles (menos eficiente que diccionario, necesito conocer
		el espacio de busqueda).
		\item Choques: Si no se utiliza SALT, a cada clave le corresponde un unico hash. Es fácil encontrar que dos usuarios
		usan la misma clave si tengo la base de datos y el espacio de búsqueda se reduce significativamente. Puedo además
		precomputar hashes y buscarlos en la base de datos de passwords $c$.
		\item Formula de Anderson:

		$$ P \geq \frac{T \cdot G}{N}$$
	
		\begin{itemize}
			\item $P$ la probabilidad de obtener una clave en un período de tiempo.
			\item $G$ el número de intentos posibles por unidad de tiempo.
			\item $T$ la cantidad de unidades de tiempo.
			\item $N$ el número de claves posibles.
		\end{itemize}

		\item Logins repetidos: Intentar darle al sistema desde afuera. Se puede enlentecer cortando despues de una
		cierta cantidad de logins, deshabilitando la cuenta o creando un ambiente \textit{sandboxeado} para el agresor.
	\end{itemize}
\end{itemize}

\subsection{Claves}

\begin{itemize}
	\item Maneras de elegirlas: \begin{itemize}
		\item Aleatorias: Las elige el sistema. Tienen casos borde (claves muy cortas, caracteres repetidos) si no se tiene
		cuidado. Son difíciles de recordar y su calidad depende de la implementación del generador de números aleatorio.
		\item Pronunciables: Basadas en generación de fonemas, lo que permite que sean fáciles de pronunciar y por ello de
		recordar. Son pocas.
		\item Elegidas por el usuario: Suelen ser ``faciles'', en el sentido que hay patrones establecidos. Se puede hacer
		chequeo activo de claves para evitar que el usuario utilice claves muy sencillas. \begin{itemize}
			\item \texttt{passwd+} provee un lenguaje de scripting para definir estas restricciones.
			\item Se puede forzar el usuario a cambiar a un password nuevo cada cierto tiempo. Avisar con anticipación para que el
			usuario pueda pensar.
		\end{itemize}
		\item Challenge response: El usuario y el sistema comparten una función secreta $f$, que puede ser una criptográfica
		basada en alguna clave compartida. \begin{itemize}
			\item El usuario pide ser autenticado.
			\item El sistema le contesta un valor $r$ aleatorio que es el ``desafío''.
			\item El usuario contesta $f(r)$, la ``respuesta''.
			\item Se puede atacar de manera similar a las claves, si dispongo de $f$,$r$ y $f(r)$.
			\item Se puede evitar cifrando el challenge $r$. 
		\end{itemize}
		\item Protocolo EKE: Asume que Alice y Bob tienen una clave $s$ que comparten. Generan una clave $k$ de sesión.

		\ig{0.3}{U5_protocoloEKE.png}

		\item One Time Passwords: Se invalidan una vez que son usados. Se envía un número y la respuesta es la clave
		asociada a ese numero. Se dificulta sincronizar las claves, es dificil generar buenas claves aleatorias y
		distribuir las claves entre los clientes y servidores como para poder hacer esto andar.
		\item One Time Passwords con Hardware: \begin{itemize}
			\item Tokens: Utilizado para calcular la respuesta a un challenge. Puede requerir de un PIN de usuario.
			\item Tiempo: Cada lapso de tiempo muestra un número. El usuario usa ese número junto con una clave fija
			adicional.
		\end{itemize}
		\item Biometría: Medición automática de características biológicas o de comportamiento de un individuo.
		\begin{itemize}
			\item Huellas digitales, mapeado a un grafo y comparado de manera aproximada con una base de datos.
			\item Voces: Verificación y reconocimiento.
			\item Ojos: Patrón único del iris, muy muy intrusivo.
			\item Cara: Detección de patrones como largo del labio, ángulo de cejas, etc.
			\item Secuencias de tipeo (presión sobre las teclas, cadencia, etc.)
		\end{itemize}
		\item Localización: Sabiendo donde esta el usuario, validar si esta ahi (usa GPS).
	\end{itemize}
\end{itemize}

\subsubsection{PAM}

\begin{itemize}
	\item \textit{Pluggable Authentication Modules}: Mecanismo flexible para la autenticación de usuarios. PAM
	permite el desarrollo de programas independientes del mecanismo de autenticación a usar. Así es posible que
	se pueda usar desde autenticación por password básica hasta mecanismos físicos o de localización.
	\item Permite distintas políticas de autenticación para cada servicio.
	\item Es configurable, utiliza un repositorio de métodos a usar con un archivo del mismo nombre que el servicio
	a usar en \texttt{/etc/pam.d/}.
	\item Módulos independientes efectuan el chequeo: \begin{itemize}
		\item Sufficient: Acepta si el módulo acepta.
		\item Required: Falla si el módulo falla, pero ejecuta todos los requisitos antes de reportar falla.
		\item Requisite: Igual a required pero no falla en el resto.
		\item Optional: Solo es invocado si todos fallan.
	\end{itemize}
\end{itemize}

\subsubsection{GINA}

\begin{itemize}
	\item Graphical Identification and Authentication Library
	\item Componente de Windows que se carga en el contexto de WinLogon. 
	\item Atiende Ctrl+Alt+Del e interactua con el usuario. Luego arranca el proceso inicial de usuario (shell).
	\item Puede ser reemplazado, hay implementaciones con biometría o tokens en hardware.
	\item Servía para Windows XP/2003
\end{itemize}

\subsubsection{MS Credentials Providers}

\begin{itemize}
	\item Nuevo mecanismo a partir de Windows Vista.
\end{itemize}

\ig{0.7}{U5_MSCredentialsProviders.png}

\subsection{Almacenamiento de claves}

\begin{itemize}
	\item Lan Manager Hash: \begin{itemize}
		\item Se convierte todo a mayúsculas antes de crear el hash.
		\item Se usa un set de characteres reducido dentro de ASCII.
		\item El hash se divide en dos bloques de 7 caractéres. Si la clave tiene menos de 14
		caractéres se paddea con null.
		\item Utiliza DES con el password "KGS!@\#\$\%".
		\item No usa SALT y el hash es un valor de 16 bytes.
		\item Muy inseguro, se puede romper fácilmente.
	\end{itemize}
	\item NT Hash: \begin{itemize}
		\item Distingue entre mayúsculas y minúsculas.
		\item Usa MD4 (inseguro).
		\item Longitud variable (no se paddea) hasta 128 caractéres.
		\item No se usa salting.
		\item NT4 lo soporta desde Service Pack 4.
	\end{itemize}
	\item Linux - MD5: \begin{itemize}
		\item Contraseña se almacena en un archivo solo accesible por root (\texttt{/etc/shadow}) y la clave se
		cifra usando tres campos separados por \$:

		\begin{itemize}
			\item El algoritmo de \textit{one way hashing} usado (1 es MD5).
			\item El SALT usado.
			\item El Hash propiamente dicho.	
		\end{itemize}

		\item Se pueden usar otros algoritmos como Blowfish o SHA-256 o SHA-512.
		\item Se puede configurar el número de repeticiones.
	\end{itemize}
	\item OpenBSD: Implementa un mecanismo por el cual el tiempo computacional necesario para cifrar una password 
	con una función de una vía puede ir variando en el tiempo, a medida que avanza la velocidad del hardware. 
	Las claves de usuarios con mayores privilegios pueden ser configuradas para ser más  difíciles de calcular.
\end{itemize}

\subsubsection{Cracking - Herramientas}

\begin{itemize}
	\item John The Ripper: Permite crackear claves MD5, Crypt, Blowfish, LM, etc. Usando diccionarios, reglas o
	fuerza bruta.
	\item Rainbow Tables: Precomputar los pares (hash, texto en claro) y usarlos para buscar de manera rápida hashes.
\end{itemize}
