\section{Unidad 5: Seguridad en Redes}

\subsection{Introducción}
Definiciones:

\begin{itemize}
	\item \textbf{TCP}: \textit{Transmission Control Protocol}, protocolo
	orientado a la conexión. Provee control de flujo, recuperación de errores
	y confiabilidad.
	\item \textbf{UDP}: \textit{User Datagram Protocol}, protocolo orientado
	a la conexión. Muy sencillo, no provee garantías. La recuperación de errores
	es responsabilidad de la aplicación.
	\item \textbf{ICMP}: \textit{Internet Control Message Protocol}. Es usado para
	mensajes de control, mensajes de error, etc. El Ping utiliza ICMP. Algunos tipos
	de ICMP:
		\begin{itemize}
			\item Echo Request
			\item Echo Reply
			\item Destination Unreachable
			\item Time Exceeded
			\item Timestamp
			\item Timestamp Reply
			\item Redirect Message
		\end{itemize}
\end{itemize}

\subsubsection{Three Way Handshake}

\ig{0.3}{U5_tcp_3way.png}

\subsubsection{Sniffers}

\begin{itemize}
	\item Un \textit{Sniffer} es un programa de software o hardware que puede ``ver'' y registrar
	el tráfico que pasa sobre una red digital. Mientras el flujo de datos viaja por la red, el
	sniffer captura cada paquete y opcionalmente lo decodifica en base a reglas, estándares y
	especificaciones.
	\item Se puede sniffear todo el tráfico que pasa por una red o solo una parte.
	\item Opera en ``modo promiscuo'' porque escucha todo lo que pasa en el medio, incluso lo que
	no va para él.
	\item Usos:
		\begin{itemize}
			\item Analizar problemas de red.
			\item Detectar intentos de intrusión a través de la red.
			\item Obtener información para luego hacer una intrusión.
			\item Monitorear el uso de la red.
			\item Reportar estadísticas de red.
			\item Espiar a otros usuarios de la red.
			\item Hacer ingeniería reversa de protocolos.
		\end{itemize}
	\item Ejemplo: TCPDump, Wireshark, etc.
\end{itemize}

\subsubsection{Monitoreo}

\begin{itemize}
	\item La mayoría de las redes \textit{switcheadas} puede definir un puerto de monitoreo, mecanismo
	conocido como \textit{port mirroring}. En este puerto se copia todo el tráfico. Sino se puede utilizar
	un \textit{network tap} insertado en el segmento de red para que transmita todo el tráfico.
	\item Se puede utilizar una estación de trabajo con dos placas de red, para armar un \textit{bridge}
	transparente. Para administrar este equipo que esta en el medio del tráfico, se puede usar una tercera
	placa con IP para accederla por afuera.
\end{itemize}

\subsubsection{ARP spoofing}

\begin{itemize}
	\item Definición: ARP (\textit{Address Resolution Protocol}) es el protocolo responsable de encontrar
	la dirección de hardware (MAC) que corresponde a una determinada IP. 
	\item Funcionamiento: Se envía un paquete del tipo ARP Request a la dirección de Broadcast conteniendo la
	dirección IP a la que se quiere contactar. Se espera un ARP Reply con la dirección MAC correspondiente.
	\item ARP Table: Cada equipo tiene una tabla temporal como \textit{cache} de resultados obtenidos.
	\item ARP Spoofing: Consiste en, con el propósito de hacerse pasar por otra una maquina con otra IP (por ejemplo
	el \textit{default gateway}), enviar paquetes ARP Reply \textit{spoofeados} a la red local para que quede asociado
	a esa IP.
\end{itemize}

\subsubsection{IP spoofing}

\begin{itemize}
	\item IP spoofing consiste en crear paquetes IP con una IP de origen distinta a la IP del que envía.
	\item El receptor confía en que la IP del paquete es la del emisor.
	\item Sirve diversos propósitos:
		\begin{itemize}
			\item Esconder el origen de un ataque.
			\item Secuestrar una sesión abierta.
			\item Aprovecharse de aplicaciones que autentifican con la IP de origen.
		\end{itemize}
\end{itemize}

\subsubsection{Ataques de Denegación de Servicio}

\begin{itemize}
	\item Definición: Un ataque de Denegación de Servicio (DoS) consiste en evitar que un sistema pueda ser usado
	por usuarios legitimos, generalmente mediante la sobrecarga de pedidos tal que no hay recursos suficientes para
	los pedidos legitimos.
	\item Ataque de SYN Flooding: 
		\begin{itemize}
			\item Cada paquete con el \textit{flag} SYN prendido crea una nueva conexión \textit{half-open}.
			\item Estas conexiones tardan minutos en \textit{timeout}ear y desaparecer.
			\item Las tablas de conexiones son finitas, con lo cual se pueden sobrecargar.
			\item Se puede \textit{spoofear} la IP.
			\item Se pueden utilizar las SYN Cookies para evitar esto.
				\begin{itemize}
					\item Las SYN Cookies son para establecer el número inicial de secuencia TCP de manera de poder saber
					que MSS esta usando la conexión y poder reconstruir esa cola SYN en su momento.
					\item Es decir, se guarda el estado en el número de secuencia y no en el servidor (disminuye la carga).
					\item Se calcula en base de las IPS, puertos, etc.
				\end{itemize}
		\end{itemize}	
	\item Reset de Conexiones:
		\begin{itemize}
			\item Un host envia un paquete RST con una IP \textit{spoofeada} a cualquier extremo de la conexión. 
			\item Necesito saber las IPs y puertos de los extremos de la conexión.
			\item Necesito tener un numero de secuencia dentro de la ventana que se esta enviando. Lo puedo obtener
			escuchando el tráfico y adivinando con los bits.
		\end{itemize}
	\item Ataque de ICMP: 
		\begin{itemize}
			\item Fabricar un paquete ICMP que indique un error ``hard''
			\begin{itemize}
				\item Protocol Unreachable
				\item Port unreachable
				\item Fragmentation Needed y DF set.
			\end{itemize}
			\item Segun el RFC 1122, TCP debería abortar la conexión.
			\item No se recomienda hacer chequeo de los errores por lo tanto no hay que adivinar números de secuencia.
		\end{itemize}
\end{itemize}

\subsubsection{Comandos varios}

\begin{itemize}
	\item Los Protocolos r (\textit{rcp}, \textit{rlogin}, \textit{rsh}, \textit{rwho}). 
	\item Permitir el acceso a terminales remotos sin tener que loguearse con usuario y clave.
	\item Los archivos \texttt{/etc/hosts.equiv} y \texttt{.rhosts} proveen el mecanismo de autenticación.
	\begin{itemize}
		\item Especifican que equipos remotos son confiables.
		\item Los usuarios confiables pueden acceder sin password.
		\item \texttt{/etc/hosts.equiv} vale para todo el sistema, cada usuario tiene su \texttt{.rhosts}
	\end{itemize}
	\item Ident: Protocolo para identificar al usuario remoto de una conexión TCP determinada.
	\begin{itemize}
		\item Permite especificar dos puertos (uno X local a la maquina A y otro remoto Y en la maquina B) para que
		B devuelva el usuario que esta conectado desde el puerto Y de B al X de A.
	\end{itemize}
	\item Telnet
	\begin{itemize}
		\item Login remoto.
		\item Tráfico en claro.
		\item User y password para loguearse.
		\item Suceptible a \textit{man in the middle}, \textit{session hijacking}, etc.
	\end{itemize}
	\item SSH
	\begin{itemize}
		\item Login remoto, transferencias de archivos, y \textit{tunneling} de conexiones.
		\item Cifra todo el tráfico para eliminar las ``escuchas''.
		\item Mecanismo de user/pass o mecanismo de pares de claves 
		\item Protege MitM, Session Hijacking, Spoofing, Sniffing, Tampering, etc.
	\end{itemize}
	\item TFTP:
	\begin{itemize}
		\item Protocolo sin autenticación para transferencia de archivos.
		\item Se puede usar para bootear discos a través de una red.
	\end{itemize}
	\item HTTP (80):
	\begin{itemize}
		\item Protocolo de aplicación para sistemas de información hipermediales, distribuidos y colaborativos.
		\item Protocolo simple, \textit{stateless}, basado en texto.
		\item Cliente envía \textit{requests} al servidor, con un método (GET,POST,etc.), URI, headers y un cuerpo opcional.
		\item 
	\end{itemize}
	\item HTTP Authentication
	\begin{itemize}
		\item Cliente hace un \textit{request}
		\item Servidor responde con código 401 y pide autenticación.
		\item Cliente envía la request, y además sus credenciales encodeadas en base 64 (no provee confidencialidad)
		\item Servidor valida las credenciales y contesta con el pedido.
	\end{itemize}
	\item HTTPS (443)
	\begin{itemize}
		\item Comunicación encriptada mediante SSL.
		\item Permite identificar fehacientemente al servidor y da confidencialidad.
		\item 
	\end{itemize}
	\item SMTP, POP3
	\begin{itemize}
		\item Se pueden usar con SSL
		\item Problema de \textit{Open Relay}: Un servidor configurado para enviar mail de cualquier dirección y a 
		cualquier dirección. Solía ser la configuración por \textit{default}. Es usado por \textit{spammers} y usualmente
		esta blacklisteada.
	\end{itemize}
	\item DNS
	\begin{itemize}
		\item \textit{Domain Name System} como esquema jerárquico de resolución de nombres.
		\item Hay servidores primarios y secundarios, que resuelven de a pedazos.
		\item Las consultas se responden preguntando quien es el servidor responsable de cada zona.
	\end{itemize}
	\item SMTP
	\begin{itemize}
		\item \textit{Simple Network Management Protocol} versión v3 sirve para administrar máquinas con ambiente no asegurado.
	\end{itemize}
\end{itemize}

\subsection{Firewalls}

\begin{itemize}
	\item \textbf{Firewall: } Es un separador entre dos partes de una red para controlar lo que pasa entre ellas. 
	\item \textbf{Propositos: }
		\begin{itemize}
			\item Proteger mis datos, a nivel de confidencialidad, integridad y disponibilidad.
			\item Proteger mis recursos de uso por agentes externos (ejemplo \textit{botnets}, \textit{zombies}, etc.).
			\item Proteger mi reputación: Evitar que mis recursos se usen para fines maliciosos generando desconfianza en mí.
		\end{itemize}
	\item \textbf{Tipos: }
		\begin{itemize}
			\item Filtrado de paquetes: Cada paquete que entra o sale de la red es verificado y permitido o denegado de acuerdo
			a un conjunto de reglas definidas por el usuario.
			\begin{itemize}
				\item Se basa en las direcciones (IP + Puerto) de origen y destino, además del protocolo.
				\item Suelen usarse en los routers.
				\item Son eficientes y fáciles de implementar.
				\item Puede producir reglas complicadas y potencialmente inseguras. Por ejemplo:
				\item FTP Activo, reglas para el servidor y cliente: \begin{itemize}
					\item S: Permitir cualquier puerto $p \in [1024,65535]$ al 21 del server FTP
					\item S: Permitir al 21 del server FTP cualquier $p \in [1024,65535]$
					\item S: Permitir al 20 del server FTP cualquier $p \in [1024,65535]$
					\item C: Permitir cualquiera en mi red de cualquier puerto alto a un puerto 21
					\item C: Permitir cualquier 21 a un puerto alto.
					\item C: Permitir cualquier 20 a un puerto alto.
					\item C: Permitir cualquiera en mi red de cualquier puerto alto a un puerto 20.
				\end{itemize}
			\end{itemize}
			\item Stateful Inspection: Es como filtrado de paquetes normal, pero además puede tener en cuenta el estado de la sesión
			de la conexión, y utilizarlas para aplicar reglas que varían según el momento de la conexión. 
			\begin{itemize}
				\item Puede analizar el protocolo superior que se esta usando.
				\item Mayor precisión en el filtrado, reglas más cortas y más estrictas.
				\item Mayor necesidad de procesamiento para cada paquete y cada conexión (potencialmente caro).
			\end{itemize}
			\item Gateways de circuito: \textit{proxy} no inteligente, simplemente renvían la conexión. Están en la capa de sesión del 
			modelo OSI\. Son independientes del protocolo pero el cliente debe conocerlo. Se usan con políticas estrictas de filtrado.
			\item Gateways de aplicación: \textit{proxy} inteligente. Conoce del protocolo con el que se esta haciendo la comunicación. El
			cliente no solo debe conocerlo sino que se debe estar usando un protocolo que permita proxys. Permite un mejor uso de autenticación,
			permite un mejor control de uso de servicios y facilita la generación de registros de auditoría y el uso de caches.
			\item Ingress/Egress Filtering: Sirve para evitar IP Spoofing. Ingress es que no permito pasar paquetes a mi red que vienen de afuera pero 
			cuya IP de origen es una IP interna. Egress es que paquetes con IP \textit{spoofeada} no pasan por el Firewall (para evitar que se
			dañe mi reputación).
			\item Personal: Es un software instalado en una computadora generalmente personal y que controla la comunicación entre el equipo y
			el mundo exterior. Permiten por ejemplo filtrar accesos por aplicación, protocolo, etc\. y consultar al usuario cuando una aplicación
			quiere hacer una conexión.
		\end{itemize}
\end{itemize}

\subsubsection{NAT}

\begin{itemize}
	\item Definición: \textit{Network Address Translation}. Consiste en rescribir las IPs de origen y destino \textit{on the fly} para permitir
	modificar las políticas de envio. Se usa por ejemplo para tener una red privada con un punto de acceso en la internet pública (ejemplo: un router
	\textit{wifi} que usa NAT para tener una sola IP pública que le da el \textit{ISP} pero para llegar a todos los dispositivos de la casa, renviandoles
	a sus IPS privadas de manera acorde).
	\item Tipos: \begin{itemize}
		\item DNAT: Consiste en modificar la IP de destino a los paquetes que llegan a una IP privada, y modificar la IP de origen cuando ocurre lo inverso.
		También conocido como \textit{port forwarding}. Se usa para publicar un servicio local en la Internet, o poner un servidor en la \textit{DMZ}.
		\item SNAT: Lo mismo que DNAT pero usando la IP de origen. El significado varía segun \textit{vendor}
		\item IP Masquerading: Consiste en que el Router enmascara la IP de una computadora local con la propia, de manera que el pedido parece venir
		de la IP indicada en la máscara. Permite ocultar la topología de la Red. Es una forma de SNAT.
	\end{itemize}
\end{itemize}

\subsubsection{Esquemas de redes}

\begin{itemize}
	\item Screening Router: Filtra paquetes a la red interna. Generalmente usan filtrado estático y reglas complejas.
	\item Bastion Host: Se utiliza un servidor en la DMZ que es el que esta más ``expuesto'' a ataques. Generalmente corre un proxy server
	y nada más, y se tunea especialmente (se lo audita seguido, corre software parcheado y que no tenga exploits conocidos, etc.) para ser lo más resistente 
	posible a ataques. Suele tener reglas simples para evitar agujeros que permitan ataques. Protege entonces el resto de la red.
	\item Screened Host: Se utiliza un Bastion Host que efectivamente SEPARA las dos redes. Usa proxies y es muy seguro. Como el Bastion Host es el que hace
	las requests entre zonas, no necesita NAT.
\end{itemize}

\subsubsection{Tipos de políticas de Firewall}

\begin{itemize}
	\item Default Permit: Permite todo salvo algunos protocolos.
	\item Default Deny: Niega todo excepto lo explicitamente permitido.
	\item Default Permit es MUY MUY INSEGURO, porque cualquier servicio habilitado sin conocimiento y que posee
	una vulnerabilidad es una puerta a un ataque.
\end{itemize}

\subsubsection{Implementación}

\begin{itemize}
	\item Diseño de la red (topología)
	\item Definición de políticas entre las redes (usando herramientas gráficas como FWBuilder o shorewall).
	\item Mantenimiento: Mantener las reglas al día, mantener el software y el SO parcheados, y revisar los logs de firewall.
\end{itemize}

\subsection{Monitoreo de redes}

\subsubsection{Recolección de trafico}

\begin{itemize}
	\item \textbf{tcpdump}: Consigue todos los paquetes, con un alto costo de almacenamiento y procesamiento pero permite mejor granularidad, con lo
	cual se pueden utilizar todas las herramientas que queramos para analizarlos y se puede hacer un análisis forense más completo, incluso en
	tráfico cifrado.
	\item \textbf{sesiones (argus,...)}: Resumen de conversaciones entre sistemas. Es compacto e independiente de los datos que se esten enviando.
	No tiene en cuenta datos de la aplicación así que no es suceptible por encriptación, utiliza los paquetes de los protocolos para armar tablas de
	conversaciones.
	\item \textbf{estadisticas (tcpdstat,...)}: Visión sumarizada de eventos, permite ver el tráfico a alto nivel. Se puede usar por ejemplo en equipos
	que generen muchisimo tráfico.
	\item \textbf{alertas (snort,...)}: Alerta por IDS tradicionales, por regla o anomalía.
\end{itemize}

\subsubsection{IDS}

\begin{itemize}
	\item Definición: Se denomina IDS (\textit{Intrusion detection system}) al proceso de monitorear los eventos que ocurren en un sistema o red de
	computadoras, buscando señales que indiquen que haya habido o esta habiendo una intrusión.
	\item Problemas comunes: \begin{itemize}
		\item \textbf{Falso positivo}: Ocurre cuando una herramienta clasifica una acción como una posible intrusión pero la misma era un uso
		legítimo del sistema, y no constituye una violación de seguridad. Se pueden disminuir haciendo más específicos los patrones a detectar.
		\item \textbf{Falso negativo}: Ocurre cuando una intrusión no es detectada por el IDS.
		\item \textbf{Falsa alarma}: Cuando se dispara una alerta por un ataque por un patrón de comportamiento que es parte de uno, pero donde el mismo
		no representa un peligro. Por ejemplo, el uso de un exploit en un software distinto al vulnerable, o que fue parcheado, o que no funciono.
	\end{itemize}
	\item \textbf{Objetivos}: \begin{itemize}
		\item Detectar una amplia variedad de intrusiones, con la correspondiente necesidad de adaptarse a nuevos ataques o patrones de comportamiento y
		aprenderlos, por parte de la herramienta.
		\item Detectar las intrusiones en un tiempo razonable, incluso a veces en tiempo real, sin afectar los tiempos de respuesta del sistema. Se puede
		por ejemplo detectar intrusiones en las ultimas horas o minutos.
		\item Presentar la información de manera fácil de entender: A veces se monitorean muchos sistemas, y el analista necesita además información sobre
		el ataque para poder determinar si fue un ataque y que respuesta dar. Lo ideal sería un indicador binario (OK - Alerta) pero no alcanza.
		\item Ser tan preciso como se pueda.
	\end{itemize}
	\item Clasificacion: \begin{itemize}
		\item IDS de Host (HIDS, ejemplo Swatch) vs IDS de Red (NIDS, ejemplo SNORT). \begin{itemize}
			\item NIDS: Reconoce patrones en el tráfico de Red.
			\item HIDS: Monitorea logs e integridad de archivos, comportamiento extraño de usuarios, parametros de sistema excedidos.
			\item Ataques a NIDS: Insertion (Meter un paquete que el IDS acepta pero el sistema no, de manera que ambos ven cosas distintas) y
			Evasion (Meter un paquete que el IDS rechaza pero que el sistema aceptaría, de manera de esconder un ataque en por ejemplo la manera en la
			que se rearman los paquetes).
		\end{itemize}
		\item Basado en heuristicas/estadisticas vs Basado en patrones
	\end{itemize}
\end{itemize}

\subsubsection{SNORT}

Arquitectura: 

\ig{0.6}{U5_arquitecturaNids.png}

Componentes:

\begin{itemize}
	\item Sniffer: Captura los paquetes de la red usando pcap.
	\item Decodificador: Identifica protocolos y construye las estructuras de datos necesarias para examinar los paquetes.
	\item Preprocesadores: Prepara los datos para detección. Se dedica a por ejemplo detectar escaneos, ensamblar los datos
	contenidos en paquetes de una misma sesión o reensamblar paquetes fragmentados.
	\item Motor de detección: Analiza los paquetes en base a reglas definidas para detectar ataques.
	\item Salida o postprocesadores: Permiten definir que, como y donde se guardan las alertas y los paquetes de red que las
	generaron.
\end{itemize}

Diferencias y cosas a remarcar

\begin{itemize}
	\item Existen herramientas como Snorby o Sguil que permiten analizar los resultados de Snort.
	\item Existen otras herramientas como Bro NSM que también pueden describir restricciones de actividad, analizar contexto,
	relacionar eventos, etc.
\end{itemize}

\subsubsection{HIDS}

\begin{itemize}
	\item Swatch: \begin{itemize}
		\item Empezo como un simple watchdog para monitorear actividades del log que producía syslog en UNIX.
		\item Permite definir entradas a resaltar o ignorar.
	\end{itemize}
	\item Logcheck: \begin{itemize}
		\item Similar a Swatch, con reglas adaptadas para Debian. Se ejecuta usualmente cada hora, revisa los logs
		por data anormal y genera un reporte que se puede mandar por email.
	\end{itemize}
	\item Tripwire,Aide: \begin{itemize}
		\item Monitorea el agregado, borrado y modificación de archivos. Genera una base de datos con la información
		de cada archivo en el sistema. Periodicamente vuelve a generar la información y la compara.
		\item Checkea archivos nuevos, eliminados, contenido y atributos (fecha, firmas, permisos, tamaño, etc.)
	\end{itemize}
	\item OSSEC \begin{itemize}
		\item HIDS Open Source, multiplataforma, realiza analisis de logs, checkeo de integridad, monitoreo de registro, etc.
	\end{itemize}
\end{itemize}

\subsubsection{IPS}

\begin{itemize}
	\item Definición: Un IPS de Red es un sistema de prevención de intrusiones, es decir que además de detectar intrusiones
	puede hacer algo al respecto. Pueden ser de red o de host.
	\item Red: Funciona como dispositivo inline en modo bridge, puede decidir filtrar el tráfico para que no llegue a destino
	si detecta una intrusión. Puede también modificar el contenido del tráfico.
	\item ModSecurity: Motor de detección y prevención de intrusiones para aplicaciones web, es un modulo de Apache. Es un 
	IPS de Host.
	\begin{itemize}
		\item Intercepta pedidos HTTP antes de que sean procesados por el host.
		\item Intercepta el cuerpo de los pedidos.
		\item Intercepta almacena y valida los archivos subidos.
		\item Realiza acciones antievasivas en forma automática.
		\item Procesa mediante un conjunto de reglas configurables.
		\item Las respuestas al cliente son también interceptadas y se analizan en base a reglas.
	\end{itemize}
\end{itemize}

\subsubsection{Honeypots}

\begin{itemize}
\item Honeypot: Es una trampa cuyo proposito es detectar, esquivar o contraatacar al mal uso de un sistema informático.
	Se suele implementar como una computadora que parece estar conectada a la red y que tiene recursos de interés para un
	atacante, pero que en realidad esta fuertemente monitoreada y aislada de la red.
\item Tipos: \begin{itemize}
		\item Baja interaccion: Emula servicios, aplicaciones, SO, etc. Bajo riesgo, faciles de mantener, pero proveen información limitada.
		\item Alta interacción: Servicios, aplicaciones y SO reales. Consumen mucho tiempo de mantener, alto riesgo pero dan mucha información.
	\end{itemize}
\item Ejemplos en orden creciente de interacción: Honeyd, Nepenthes, Honeynets
\item Honeyd: \begin{itemize}
	\item Crea equipos virtuales en una red
	\item Los equipos puede ser configurados para ejecutar servicios arbitrarios.
	\item Simula distintos sistemas operativos.
	\item Puede reportar resultados a Prelude.
	\end{itemize}
\item Nepenthes:\begin{itemize}
	\item Baja interacción, emula vulnerabilidades conocidas, especialmente las que usan los worms para diseminarse y 
	capturar worms.
	\end{itemize}
\item Honeynets: \begin{itemize}
	\item Se agregan equipos enteros, no un producto o software. De alta interacción. Toda la red es altamente monitoreada y
		los paquetes son capturados.
	\item Control de flujo de datos:  Es necesario controlar los datos por el riesgo de que la honeynet sea usada como punto de ataque.
	\item Captura de datos: Se toma la actividad de red, de sistema y de aplicaciones.
	\item Análisis de datos.
	\item Sebek: Herramienta de captura de datos que corre como módulo de kernel. Envia la información de la red sobre el atacante
		a un servidor de manera que no se pueda sniffear, utilizando para eso SSH entre los dos. De esta manera se juntan datos sin
		que el atacante lo sepa.
	\end{itemize}
\end{itemize}

